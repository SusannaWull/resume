%%%%%%%%%%%%%%%%%%%%%%%%%%%%%%%%%%%%%%%%%
% Plasmati Graduate CV
% LaTeX Template
% Version 1.0 (24/3/13)
%
% This template has been downloaded from:
% http://www.LaTeXTemplates.com
%
% Original author:
% Alessandro Plasmati (alessandro.plasmati@gmail.com)
%
% License:
% CC BY-NC-SA 3.0 (http://creativecommons.org/licenses/by-nc-sa/3.0/)
%
% Important note:
% This template needs to be compiled with XeLaTeX.
% The main document font is called Fontin and can be downloaded for free
% from here: http://www.exljbris.com/fontin.html
%
%%%%%%%%%%%%%%%%%%%%%%%%%%%%%%%%%%%%%%%%%

%----------------------------------------------------------------------------------------
%	PACKAGES AND OTHER DOCUMENT CONFIGURATIONS
%----------------------------------------------------------------------------------------

\documentclass[a4paper,11pt]{article} % Default font size and paper size

\usepackage{fontspec} % For loading fonts
\defaultfontfeatures{Mapping=tex-text}
\setmainfont{TeX Gyre Pagella} % Main document font

\usepackage{xunicode,xltxtra,url,parskip} % Formatting packages

\usepackage[usenames,dvipsnames]{xcolor} % Required for specifying custom colors

\usepackage[big]{layaureo} % Margin formatting of the A4 page, an alternative to layaureo can be \usepackage{fullpage}
% To reduce the height of the top margin uncomment: \addtolength{\voffset}{-1.3cm}

\usepackage{hyperref} % Required for adding links	and customizing them
\definecolor{linkcolour}{rgb}{0,0.2,0.6} % Link color
\hypersetup{colorlinks,breaklinks,urlcolor=linkcolour,linkcolor=linkcolour} % Set link colors throughout the document

\usepackage{titlesec} % Used to customize the \section command
\titleformat{\section}{\Large\scshape\raggedright}{}{0em}{}[\titlerule] % Text formatting of sections
\titlespacing{\section}{0pt}{8pt}{3pt} % Spacing around sections

\usepackage{longtable}
\usepackage{array}
\newcolumntype{R}[1]{>{\raggedleft\let\newline\\\arraybackslash\hspace{0pt}}p{#1}}

\usepackage{fancyhdr}
\pagestyle{fancy}
\fancyhf{}
\lfoot{Weipeng He} 
\rfoot{Page \thepage} 
\renewcommand{\headrulewidth}{0pt}
\renewcommand{\footrulewidth}{1pt}

\hyphenation{Tsinghua}

\begin{document}

%----------------------------------------------------------------------------------------
%	NAME AND CONTACT INFORMATION
%----------------------------------------------------------------------------------------

\par{\centering{\Huge Weipeng \textsc{He}}\bigskip\par} % Your name

\section{Personal Data}

\begin{tabular}{rl}
\textsc{Place and Date of Birth:} & China | 13 November 1988 \\
\textsc{Address:} & Hagenbeckstr. 60, 22527 Hamburg, Germany \\
\textsc{Phone:} & +49 157 79527652\\
\textsc{Email:} & \href{mailto:2he@informatik.uni-hamburg.de}{2he@informatik.uni-hamburg.de}
\end{tabular}

%----------------------------------------------------------------------------------------
%	EDUCATION
%----------------------------------------------------------------------------------------

\section{Education}
\begin{tabular}{R{3.5cm}|p{11cm}}	
  \textsc{Oct 2012 - Sep 2015} & Master of Science in \textsc{Intelligent Adaptive Systems} \\
  {\footnotesize (expected)} & \textbf{University of Hamburg} \hfill Hamburg, Germany \\
  & \textsc{Gpa}: 1.50 (very good, German university grade system) \\
  \multicolumn{2}{c}{} \\

%------------------------------------------------

  \textsc{Aug 2008 - Jul 2012} & Bachelor in \textsc{Computer Science and Technology} \\
  \textsc{Aug 2009 - Jul 2012} & Bachelor in \textsc{Mathematics} (second major) \\
  & \textbf{Tsinghua University} \hfill Beijing, China \\
  & \textsc{Gpa}: 90/100 \\

\end{tabular}

%----------------------------------------------------------------------------------------
% Experiences
%----------------------------------------------------------------------------------------

\section{Experiences}
\begin{longtable}{R{3.5cm}|p{11cm}}	
  \textsc{Oct 2012} - Present & \textsc{Technical Aspects of Multimodal Systems Group}, University of Hamburg \\
  & {\footnotesize Advisor:} Prof. Dr. Jianwei \textsc{Zhang} \\
  & {\footnotesize (Master Thesis) \textbf{Visual-Audio Object Recognition using Hidden Markov Models}.} \\
  & {\footnotesize This thesis explored novel methods for object recognition from visual and audio information. Feature fusion method and decision fusion method are studied. Experiments with 33 household objects showed that both fusion methods outperformed unimodal methods. The recognition system consists of audio and visual feature extraction modules, a hidden Markov model library and a classification module.} \\
  \multicolumn{2}{c}{} \\

%------------------------------------------------

  \textsc{Jun 2012 - Sep 2012} & Intern at \textsc{Hulu (Beijing) Software Corporation} \\
  & {\footnotesize Improved the search engine with \textbf{fast approximate string matching}.} \\
  \multicolumn{2}{c}{} \\

%------------------------------------------------

  \textsc{Feb 2012 - Jul 2012} & \textsc{Intelligent Information Acquisition Group}, Tsinghua University \\
  & {\footnotesize Advisor:} Prof. Xiaoyan \textsc{Zhu} \\
  & {\footnotesize (Bachelor Thesis) \textbf{Knowledge-Based Dialog System using Reinforcement Learning}.} \\
  & {\footnotesize Implemented a dialog system for assisting customers to choose cell phones. The system uses reinforcement learning (Q-Learing) to learn how to make dialogs more effective.} \\
  \multicolumn{2}{c}{} \\

%------------------------------------------------

  \textsc{Feb 2012 - Jul 2012} & \textsc{Department of Mathematical Sciences}, Tsinghua University \\
  & {\footnotesize Advisor:} Dr. Hao \textsc{Wu} \\
  & {\footnotesize (Bachelor Thesis) \textbf{A Study of Lagrangian Method for the Wave Equations.}} \\
  \multicolumn{2}{c}{} \\

%------------------------------------------------

  \textsc{Apr 2010 - Jan 2012} & \textsc{Nature Language Processing Group}, Tsinghua University \\
  & {\footnotesize Advisor:} Prof. Maosong \textsc{Sun} \\
  & {\footnotesize Implemented a \textbf{keyword extraction} and \textbf{word sense disambiguation} system using Wikipedia knowledge. Constructed a news corpus from Internet resources with labeled data for keyword extraction.} \\

\end{longtable}

\section{Publications}
\begin{itemize}
  \item (Accepted) "Multimodal object recognition from visual and audio sequences." \\
    Weipeng He; Haojun Guan; Jianwei Zhang, \textit{International Conference on Multisensor Fusion and Information Integration for Intelligent Systems (MFI)} , Sept. 2015 

  \item "What to do first: The initial behavior in a multi-sensory household object recognition and categorization system." \\
    Haojun Guan; Weipeng He; Jianwei Zhang, \textit{International Conference on Multisensor Fusion and Information Integration for Intelligent Systems (MFI)} , Sept. 2014

  \item "Thunlp at tac kbp 2011 in entity linking." \\
    Yu Zhao; Weipeng He; Zhiyuan Liu; Maosong Sun. \textit{Proceedings of TAC}, 2011
\end{itemize}

%----------------------------------------------------------------------------------------
%	Awards
%----------------------------------------------------------------------------------------

\section{Awards}

\begin{tabular}{rl}
  \textsc{Oct 2013 - Mar 2015} & \textbf{Merit Scholarship for International Students}, {\small University of Hamburg}. \\
  \textsc{July 2012} & \textbf{Outstanding Graduate}, {\small Tsinghua University}. \\
  & {\footnotesize (Awarded to the top 10 percents of all graduates.)}

\end{tabular}

%----------------------------------------------------------------------------------------
%	TECHNICAL SKILLS 
%----------------------------------------------------------------------------------------

\section{Technical Skills}

\begin{tabular}{rl}
  Programming Languages: & \textbf{C} $\cdotp$ C++ $\cdotp$ Java $\cdotp$ Python $\cdotp$ Matlab $\cdotp$ Scheme. \\
  Libraries: & Gstreamer $\cdotp$ OpenCV $\cdotp$ GSL. \\
  Parallel Programming: & OpenMP. \\
  Other Tools: & Vim $\cdotp$ Bash scripts $\cdotp$ \LaTeX $\cdotp$ GDB $\cdotp$ Git. \\
\end{tabular}

%----------------------------------------------------------------------------------------
%	LANGUAGES
%----------------------------------------------------------------------------------------

\section{Languages}

\begin{tabular}{rl}
  \textsc{English:} & TOEFL ibt 106 (reading 29 $\cdotp$ listening 27 $\cdotp$ speaking 23 $\cdotp$ writing 27) \\
  \textsc{Chinese:} & Native Speaker \\
  \textsc{German:} & Basic \\
\end{tabular}

\section{Miscellaneous}

\begin{tabular}{rl}
  Interests: & Traveling $\cdotp$ Handball $\cdotp$ Rowing \\
  GitHub: & \href{https://github.com/hwp}{github.com/hwp} \\
\end{tabular}

\iffalse
\section{References}

\begin{tabular}{rl}
 Prof. Dr. Jianwei \textsc{Zhang} & University of Hamburg \\
 \footnotesize{(Thesis Advisor)} & Department of Informatics, Group TAMS \\
   & Vogt-Kölln-Str. 30 \\
   & 22527 Hamburg, Germany \\
   & \href{mailto:zhang@informatik.uni-hamburg.de}{zhang@informatik.uni-hamburg.de} \\
   \\
 Prof. Dr. Stefan \textsc{Wermter} & University of Hamburg \\
   & Department of Informatics, Group WTM\\
   & Vogt-Kölln-Str. 30 \\
   & 22527 Hamburg, Germany \\
   & \href{mailto:wermter@informatik.uni-hamburg.de}{wermter@informatik.uni-hamburg.de} \\
   \\
 Dr. Andreas \textsc{M\"ader} & University of Hamburg \\
   & Department of Informatics, Group TAMS\\
   & Vogt-Kölln-Str. 30 \\
   & 22527 Hamburg, Germany \\
   & \href{mailto:maeder@informatik.uni-hamburg.de}{maeder@informatik.uni-hamburg.de} \\
\end{tabular}
\fi

\vspace{.5cm}
\par{\centering \scriptsize Last updated: \today \par}

\iffalse
\newpage

\par{\centering \Large \hypertarget{synopsis}{A Brief Synopsis of Master Thesis}\par}

Title: \textbf{Visual-Audio Object Recognition using Hidden Markov Models}

Comparing to traditional visual object recognition, multimodal object recognition is advantageous in that different modalities provide complementary information. This work aims to implement a system for object recognition given videos of interactions with objects and investigate different modality fusion methods.

The bag-of-words model with SIFT descriptors and the MFCC are used as visual and audio features. The system classify objects by computing the probabilities with learned hidden Markov models. The system incorporates two different fusion methods: feature fusion and decision fusion. The former method learns a joint probability distribution with one HMM, while the latter method learns two separate distributions for each modality and combine them under the conditional independence assumption.

Experiments based on a dataset of 33 different household objects are carried out to evaluate the performance of these two fusion methods as well as unimodal approaches. The result shows that both fusion methods outperform unimodal methods, while these two methods are mostly comparable. 
\fi
 
\end{document}

