%%%%%%%%%%%%%%%%%%%%%%%%%%%%%%%%%%%%%%%%%
% Plasmati Graduate CV
% LaTeX Template
% Version 1.0 (24/3/13)
%
% This template has been downloaded from:
% http://www.LaTeXTemplates.com
%
% Original author:
% Alessandro Plasmati (alessandro.plasmati@gmail.com)
%
% License:
% CC BY-NC-SA 3.0 (http://creativecommons.org/licenses/by-nc-sa/3.0/)
%
% Important note:
% This template needs to be compiled with XeLaTeX.
% The main document font is called Fontin and can be downloaded for free
% from here: http://www.exljbris.com/fontin.html
%
%%%%%%%%%%%%%%%%%%%%%%%%%%%%%%%%%%%%%%%%%

%----------------------------------------------------------------------------------------
%	PACKAGES AND OTHER DOCUMENT CONFIGURATIONS
%----------------------------------------------------------------------------------------

\documentclass[a4paper,11pt]{article} % Default font size and paper size

%\usepackage{fontspec} % For loading fonts
%\defaultfontfeatures{Mapping=tex-text}
%\setmainfont{TeX Gyre Pagella} % Main document font

\usepackage{url,parskip} % Formatting packages

\usepackage[usenames,dvipsnames]{xcolor} % Required for specifying custom colors

\usepackage{fullpage}


\usepackage{hyperref} % Required for adding links	and customizing them
\definecolor{linkcolour}{rgb}{0,0.2,0.6} % Link color
\hypersetup{colorlinks,breaklinks,urlcolor=linkcolour,linkcolor=linkcolour} % Set link colors throughout the document

\usepackage{titlesec} % Used to customize the \section command
\titleformat{\section}{\Large\bfseries\raggedright}{}{0em}{}[\titlerule] % Text formatting of sections
\titlespacing{\section}{0pt}{5pt}{-3pt} % Spacing around sections

\usepackage{fancyhdr}
\pagestyle{fancy}
\fancyhf{}
\rhead{Weipeng He, Page \thepage} 
\renewcommand{\headrulewidth}{0pt}
\renewcommand{\footrulewidth}{0pt}
\setlength{\headheight}{1em}

\usepackage{enumitem}
\setlist{nosep}

\hyphenation{Tsinghua}
\newcommand{\ind}{\hspace*{1em}}

\setlength{\parskip}{12pt}

\begin{document}
\thispagestyle{empty}

%----------------------------------------------------------------------------------------
%	NAME AND CONTACT INFORMATION
%----------------------------------------------------------------------------------------

\begin{center}
  {\Huge\bfseries Weipeng He}

  G7-3-701, CUMT (Wenchang), Xuzhou, 221008 China \\
  \href{mailto:heweipeng@gmail.com}{\texttt{heweipeng@gmail.com}} $\cdotp$
  \texttt{+86 15120002979}
\end{center}

%----------------------------------------------------------------------------------------
%	EDUCATION
%----------------------------------------------------------------------------------------

\section{EDUCATION}
\textbf{University of Hamburg} \hfill Hamburg, Germany \\
\ind M.Sc. in \textit{Intelligent Adaptive Systems}, GPA: 1.38/1.0 (excellent)  \hfill September 2015 \\
\ind - Thesis: Visual-Audio Object Recognition Using Hidden Markov Models (Grade: 1.0)

\textbf{Tsinghua University} \hfill Beijing, China \\
\ind B.E. in \textit{Computer Science and Technology}, GPA: 90/100 \hfill July 2012 \\
\ind B.S. in \textit{Pure and Applied Mathematics} (second major), GPA: 83/100 \hfill July 2012

%----------------------------------------------------------------------------------------
% Experiences
%----------------------------------------------------------------------------------------

\section{RESEARCH EXPERIENCE}

\textbf{University of Hamburg}, TAMS Group \hfill Hamburg, Germany \\
\textit{Research Assistant} with Professor Jianwei Zhang \hfill October 2012 - present
\vspace{-\parskip}
\begin{itemize}
  \item Developed visual-audio object recognition system using hidden Markov models.
  \item Examined and compared feature fusion and decision fusion methods for multimodal object recognition with experiments.
  \item Implemented C library for hidden Markov models (available under LGPL, \href{https://github.com/hwp/notGHMM}{website}).
  \item Tested noise reduction methods for robots' voice command using spectral subtraction and blind source separation techniques.
\end{itemize}

%------------------------------------------------

\textbf{Hulu} \hfill Beijing, China \\
\textit{Research Intern} \hfill July - September 2012
\vspace{-\parskip}
\begin{itemize}
  \item Improved search engine with fast approximate string matching.
\end{itemize}


%------------------------------------------------

\textbf{Tsinghua University}, Intelligent Information Acquisition Group \hfill Beijing, China \\
\textit{Research Assistant} with Professor Xiaoyan Zhu \hfill February - June 2012
\vspace{-\parskip}
\begin{itemize}
  \item Implemented cell phone retail dialog system incorporating reinforcement learning.
\end{itemize}

%------------------------------------------------

\textbf{Tsinghua University}, Department of Mathematical Sciences \hfill Beijing, China \\
\textit{Research Assistant} with Dr. Hao Wu \hfill February - June 2012
\vspace{-\parskip}
\begin{itemize}
  \item Studied Lagrangian method for wave equations.
\end{itemize}

%------------------------------------------------

\textbf{Tsinghua University}, Nature Language Processing Lab \hfill Beijing, China \\
\textit{Research Assistant} with Professor Maosong Sun \hfill 2010 - 2012
\vspace{-\parskip}
\begin{itemize}
  \item Explored methods for keyword extraction and word sense disambiguation using Wikipedia data.
  \item Constructed news corpus from Internet resources for keyword extraction.
\end{itemize}

%----------------------------------------------------------------------------------------
%	Awards
%----------------------------------------------------------------------------------------

\section{SCHOLARSHIPS AND AWARDS}

Merit Scholarship for International Students, University of Hamburg \hfill 2013 - 2015 \\
Outstanding Graduate, Tsinghua University (top 10\%) \hfill July 2012 \\
Tung OOCL Scholarship, Tsinghua University \hfill December 2010  \\
Zheng Geru Scholarship, Tsinghua University \hfill December 2009

%----------------------------------------------------------------------------------------
%	TECHNICAL SKILLS 
%----------------------------------------------------------------------------------------

\section{TECHNICAL SKILLS}

\begin{tabular}{rl}
  Programming Languages: & C $\cdotp$ C++ $\cdotp$ MATLAB $\cdotp$ Java $\cdotp$ Python $\cdotp$ OCaml. \\
  Libraries: & Gstreamer $\cdotp$ OpenCV $\cdotp$ GSL. \\
  Parallel Programming: & OpenMP. \\
  Other Tools: & Vim $\cdotp$ Bash $\cdotp$ \LaTeX $\cdotp$ GDB $\cdotp$ Git $\cdotp$ Gnuplot. \\
\end{tabular}

%----------------------------------------------------------------------------------------
%	PUBLICATIONS
%----------------------------------------------------------------------------------------

\section{CONFERENCE PAPERS}
\begin{itemize}
  \item ``Multimodal object recognition from visual and audio sequences.'' \\
    Weipeng He; Haojun Guan; Jianwei Zhang \\
    \textit{2015 IEEE International Conference on Multisensor Fusion and Information Integration for Intelligent Systems (MFI)}, Sept. 2015 

  \item ``What to do first: The initial behavior in a multi-sensory household object recognition and categorization system.'' \\
    Haojun Guan; Weipeng He; Jianwei Zhang \\
    \textit{2014 IEEE International Conference on Multisensor Fusion and Information Integration for Intelligent Systems (MFI)} , Sept. 2014

  \item ``Thunlp at tac kbp 2011 in entity linking.'' \\
    Yu Zhao; Weipeng He; Zhiyuan Liu; Maosong Sun \\
    \textit{Proceedings of TAC}, 2011
\end{itemize}

%----------------------------------------------------------------------------------------
%	LANGUAGES
%----------------------------------------------------------------------------------------

\section{LANGUAGES}

\begin{tabular}{rl}
  English: & TOEFL ibt 106 (reading 29 $\cdotp$ listening 27 $\cdotp$ speaking 23 $\cdotp$ writing 27) \\
  Chinese: & Native Speaker \\
  German: & Basic \\
\end{tabular}

\section{MISCELLANEOUS}

\begin{tabular}{rl}
  Interests: & Traveling $\cdotp$ Handball \\
  GitHub: & \href{https://github.com/hwp}{github.com/hwp} \\
\end{tabular}

\iffalse
\section{REFERENCES}

\begin{tabular}{rl}
  Prof. Dr. Jianwei Zhang & University of Hamburg \\
  \footnotesize{(Thesis Advisor)} & Department of Informatics, Group TAMS \\
  & Vogt-Kölln-Str. 30 \\
  & 22527 Hamburg, Germany \\
  & \href{mailto:zhang@informatik.uni-hamburg.de}{zhang@informatik.uni-hamburg.de} \\
  \\
  Prof. Dr. Stefan Wermter & University of Hamburg \\
  & Department of Informatics, Group WTM\\
  & Vogt-Kölln-Str. 30 \\
  & 22527 Hamburg, Germany \\
  & \href{mailto:wermter@informatik.uni-hamburg.de}{wermter@informatik.uni-hamburg.de} \\
  \\
  Dr. Andreas M\"ader & University of Hamburg \\
  & Department of Informatics, Group TAMS\\
  & Vogt-Kölln-Str. 30 \\
  & 22527 Hamburg, Germany \\
  & \href{mailto:maeder@informatik.uni-hamburg.de}{maeder@informatik.uni-hamburg.de} \\
\end{tabular}
\fi

\vspace{.5cm}
\par{\centering \scriptsize Last updated: \today \par}

\iffalse
\newpage

\par{\centering \Large \hypertarget{synopsis}{A Brief Synopsis of Master Thesis}\par}

Title: \textbf{Visual-Audio Object Recognition using Hidden Markov Models}

Comparing to traditional visual object recognition, multimodal object recognition is advantageous in that different modalities provide complementary information. This work aims to implement a system for object recognition given videos of interactions with objects and investigate different modality fusion methods.

The bag-of-words model with SIFT descriptors and the MFCC are used as visual and audio features. The system classify objects by computing the probabilities with learned hidden Markov models. The system incorporates two different fusion methods: feature fusion and decision fusion. The former method learns a joint probability distribution with one HMM, while the latter method learns two separate distributions for each modality and combine them under the conditional independence assumption.

Experiments based on a dataset of 33 different household objects are carried out to evaluate the performance of these two fusion methods as well as unimodal approaches. The result shows that both fusion methods outperform unimodal methods, while these two methods are mostly comparable. 
\fi

\end{document}

