%%%%%%%%%%%%%%%%%%%%%%%%%%%%%%%%%%%%%%%%%
% Plasmati Graduate CV
% LaTeX Template
% Version 1.0 (24/3/13)
%
% This template has been downloaded from:
% http://www.LaTeXTemplates.com
%
% Original author:
% Alessandro Plasmati (alessandro.plasmati@gmail.com)
%
% License:
% CC BY-NC-SA 3.0 (http://creativecommons.org/licenses/by-nc-sa/3.0/)
%
% Important note:
% This template needs to be compiled with XeLaTeX.
% The main document font is called Fontin and can be downloaded for free
% from here: http://www.exljbris.com/fontin.html
%
%%%%%%%%%%%%%%%%%%%%%%%%%%%%%%%%%%%%%%%%%

%%%%%%%%%%%%%%%%%%%%%%%%%%%%%%%%%%%%%%%%%%%%%%%%%%%%%%%%%%%%%
%	PACKAGES AND OTHER DOCUMENT CONFIGURATIONS
%%%%%%%%%%%%%%%%%%%%%%%%%%%%%%%%%%%%%%%%%%%%%%%%%%%%%%%%%%%%%

\documentclass[a4paper,11pt]{article} % Default font size and paper size

\usepackage{fontspec} % For loading fonts
%\defaultfontfeatures{Mapping=tex-text}
\setmainfont{Lora}
\setsansfont{Muli}
\setmonofont{Muli}
%\setmonofont[Scale=0.9]{Roboto Mono}

\usepackage{url,parskip} % Formatting packages

\usepackage[usenames,dvipsnames]{xcolor} % Required for specifying custom colors

\usepackage{fullpage}


\usepackage{hyperref} % Required for adding links	and customizing them
\definecolor{linkcolour}{rgb}{0,0.2,0.6} % Link color
\hypersetup{colorlinks,breaklinks,urlcolor=linkcolour,linkcolor=linkcolour} % Set link colors throughout the document

\usepackage{titlesec} % Used to customize the \section command
\titleformat{\section}{\Large\bfseries\scshape\sffamily\raggedright}{}{0em}{}[\titlerule] % Text formatting of sections
\titlespacing{\section}{0pt}{5pt}{-3pt} % Spacing around sections

\usepackage{fancyhdr}
\pagestyle{fancy}
\fancyhf{}
\rhead{Weipeng He, Page \thepage} 
\renewcommand{\headrulewidth}{0pt}
\renewcommand{\footrulewidth}{0pt}
\setlength{\headheight}{1em}

\usepackage{enumitem}
\setitemize{itemsep=-.4em,leftmargin=2em}
\renewcommand\labelitemi{-}

\hyphenation{Tsinghua}
\newcommand{\ind}{\hspace*{1em}}

\setlength{\parskip}{12pt}

\begin{document}
\thispagestyle{empty}

%%%%%%%%%%%%%%%%%%%%%%%%%%%%%%%%%%%%%%%%%%%%%%%%%%%%%%%%%%%%%
%	NAME AND CONTACT INFORMATION
%%%%%%%%%%%%%%%%%%%%%%%%%%%%%%%%%%%%%%%%%%%%%%%%%%%%%%%%%%%%%

\begin{center}
  {\Huge\bfseries\sffamily Weipeng He}

  \sffamily Idiap Research Institute, Rue Marconi 19, 1920 Martigny, Switzerland \\
  \url{https://idiap.ch/~whe} $\cdotp$
  \href{mailto:heweipeng@gmail.com}{\texttt{heweipeng@gmail.com}} $\cdotp$
  \texttt{+41 27 721 77 71}
\end{center}

%%%%%%%%%%%%%%%%%%%%%%%%%%%%%%%%%%%%%%%%%%%%%%%%%%%%%%%%%%%%%
%	Research Interests
%%%%%%%%%%%%%%%%%%%%%%%%%%%%%%%%%%%%%%%%%%%%%%%%%%%%%%%%%%%%%

\section{Research Interests}
My objective is research into acoustic scene analysis in complicated environments using deep learning-based methods. I am interested in sound source localization, speech enhancement, audio-visual fusion, speaker recognition, machine learning, etc. 

%%%%%%%%%%%%%%%%%%%%%%%%%%%%%%%%%%%%%%%%%%%%%%%%%%%%%%%%%%%%%
%	EDUCATION
%%%%%%%%%%%%%%%%%%%%%%%%%%%%%%%%%%%%%%%%%%%%%%%%%%%%%%%%%%%%%

\section{Education}
\textbf{\'Ecole Polytechnique F\'ed\'erale de Lausanne (EPFL)} \hfill Switzerland \\
\ind Ph.D. Candidate \hfill 2020 (expected) \\
\ind - Thesis (working title): Audio-Visual Tracking for Multi-Party Human-Robot Interaction \\
\ind - Supervisors : Dr. Jean-Marc Odobez and Dr. Petr Motlicek

\textbf{University of Hamburg} \hfill Germany \\
\ind M.Sc.\ in \textit{Intelligent Adaptive Systems}, GPA\@: 1.38/1.0 (excellent)  \hfill September 2015
%\ind - Thesis: Visual-Audio Object Recognition Using Hidden Markov Models (Grade: 1.0)

\textbf{Tsinghua University} \hfill China \\
\ind B.E. in \textit{Computer Science and Technology}, GPA\@: 90/100 (top 10\%) \hfill July 2012 \\
\ind B.S. in \textit{Pure and Applied Mathematics} (second major), GPA\@: 83/100 %\hfill July 2012

%%%%%%%%%%%%%%%%%%%%%%%%%%%%%%%%%%%%%%%%%%%%%%%%%%%%%%%%%%%%%
% Experiences
%%%%%%%%%%%%%%%%%%%%%%%%%%%%%%%%%%%%%%%%%%%%%%%%%%%%%%%%%%%%%

\section{Experience}

\textbf{Idiap Research Institute}  \hfill Switzerland \\
\textit{Research Assistant} \hfill June 2016 - Present
\vspace{-.9\parskip}
\begin{itemize}
  \item Investigated deep learning-based methods for sound source localization in real human robot interaction scenarios. Achieved more than 90\% precision and recall for detecting and localizing multiple simultaneous sources in real-time.
    [\href{https://www.youtube.com/watch?v=_4EwuVlE_pU}{video}]
  \item Investigated methods for joint localization and speech/non-speech classification of multiple sound sources.
    [\href{https://www.youtube.com/watch?v=O7bQvg03RTc}{video}]
\end{itemize}

%------------------------------------------------

\textbf{National University of Singapore}  \hfill Singapore \\
\textit{Software Engineer} \hfill January - May 2016
\vspace{-.9\parskip}
\begin{itemize}
  \item Developed web interface (back-end) of a business insight information retrieval system.
\end{itemize}

%------------------------------------------------

\textbf{University of Hamburg} \hfill Germany \\
\textit{Research Assistant}  \hfill October 2012 - November 2015
\vspace{-.9\parskip}
\begin{itemize}
  \item Studied interactive object recognition methods with audio-visual sensory fusion.
\end{itemize}

%------------------------------------------------

\textbf{Hulu (Beijing)} \hfill China \\
\textit{Research Intern} \hfill July - September 2012
\vspace{-.9\parskip}
\begin{itemize}
  \item Optimized search engine for fast and error tolerant search with fast approximate string matching. 
\end{itemize}

%------------------------------------------------

%%%%%%%%%%%%%%%%%% DELETE %%%%%%%%%%%%%%%%%%%%%%%
\iffalse  
%%%%%%%%%%%%%%%%%% DELETE %%%%%%%%%%%%%%%%%%%%%%%

\textbf{Tsinghua University}, Intelligent Information Acquisition Group \hfill Beijing, China \\
\textit{Research Assistant} \hfill February - June 2012
\vspace{-\parskip}
\begin{itemize}
  \item Implemented cell phone retail dialog system incorporating reinforcement learning.
\end{itemize}

%------------------------------------------------

\textbf{Tsinghua University}, Department of Mathematical Sciences \hfill Beijing, China \\
\textit{Research Assistant} \hfill February - June 2012
\vspace{-\parskip}
\begin{itemize}
  \item Studied Lagrangian method for wave equations.
\end{itemize}

%------------------------------------------------

\textbf{Tsinghua University}, Nature Language Processing Lab \hfill Beijing, China \\
\textit{Research Assistant} \hfill 2010 - 2012
\vspace{-\parskip}
\begin{itemize}
  \item Explored methods for keyword extraction and word sense disambiguation using Wikipedia data.
  \item Constructed news corpus from Internet resources for keyword extraction.
\end{itemize}

%%%%%%%%%%%%%%%%%%%%%%%%%%%%%%%%%%%%%%%%%%%%%%%%%%%%%%%%%%%%%
%	Awards
%%%%%%%%%%%%%%%%%%%%%%%%%%%%%%%%%%%%%%%%%%%%%%%%%%%%%%%%%%%%%

\section{Scholarships and Awards}

Merit Scholarship for International Students, University of Hamburg \hfill 2013 - 2015 \\
Outstanding Graduate, Tsinghua University (top 10\%) \hfill July 2012 \\
Tung OOCL Scholarship, Tsinghua University \hfill December 2010  \\
Zheng Geru Scholarship, Tsinghua University \hfill December 2009

%%%%%%%%%%%%%%%%%% END    %%%%%%%%%%%%%%%%%%%%%%%
\fi
%%%%%%%%%%%%%%%%%% END    %%%%%%%%%%%%%%%%%%%%%%%

%%%%%%%%%%%%%%%%%%%%%%%%%%%%%%%%%%%%%%%%%%%%%%%%%%%%%%%%%%%%%
%	PUBLICATIONS
%%%%%%%%%%%%%%%%%%%%%%%%%%%%%%%%%%%%%%%%%%%%%%%%%%%%%%%%%%%%%

\section{Publications}
\begin{itemize}
  \item ``Joint Localization and Classification of Multiple Sound Sources Using a Multi-task Neural Network.''
    Weipeng He, Petr Motlicek and Jean-Marc Odobez,
    \textit{Proc. Interspeech 2018}
    (\textbf{nominated for a best paper award})
    [\href{http://doi.org/10.21437/Interspeech.2018-1269}{doi}, \href{https://www.youtube.com/watch?v=O7bQvg03RTc}{video}]

  \item ``Deep Neural Networks for Multiple Speaker Detection and Localization.''
    Weipeng He, Petr Motlicek and Jean-Marc Odobez,
    \textit{2018 IEEE International Conference on Robotics and Automation (ICRA)}
    [\href{http://doi.org/10.1109/ICRA.2018.8461267}{doi}, \href{https://www.youtube.com/watch?v=_4EwuVlE_pU}{video}]

  \item ``Multimodal Object Recognition From Visual and Audio Sequences.''
    Weipeng He, Haojun Guan and Jianwei Zhang,
    \textit{2015 IEEE International Conference on Multisensor Fusion and Information Integration for Intelligent Systems (MFI)}

  \item ``What to Do First: The Initial Behavior in a Multi-sensory Household Object Recognition and Categorization System.''
    Haojun Guan, Weipeng He and Jianwei Zhang,
    \textit{2014 IEEE International Conference on Multisensor Fusion and Information Integration for Intelligent Systems (MFI)}

  \item ``THUNLP at TAC KBP 2011 in Entity Linking.''
    Yu Zhao, Weipeng He, Zhiyuan Liu and Maosong Sun,
    \textit{Proceedings of TAC, 2011}
\end{itemize}

%%%%%%%%%%%%%%%%%%%%%%%%%%%%%%%%%%%%%%%%%%%%%%%%%%%%%%%%%%%%%
%	TECHNICAL SKILLS 
%%%%%%%%%%%%%%%%%%%%%%%%%%%%%%%%%%%%%%%%%%%%%%%%%%%%%%%%%%%%%

\section{Technical Skills}

\begin{tabular}{rl}
  Programming: & Python $\cdotp$ C $\cdotp$ C++ $\cdotp$ Java $\cdotp$ MATLAB \\
  Libraries: & PyTorch $\cdotp$ Gstreamer $\cdotp$ OpenCV $\cdotp$ GSL $\cdotp$ OpenMP $\cdotp$ OpenMPI \\
  Other Tools: & Vim $\cdotp$ Bash $\cdotp$ \LaTeX{} $\cdotp$ GDB $\cdotp$ Git $\cdotp$ Gnuplot \\
\end{tabular}

%%%%%%%%%%%%%%%%%%%%%%%%%%%%%%%%%%%%%%%%%%%%%%%%%%%%%%%%%%%%%
%	LANGUAGES
%%%%%%%%%%%%%%%%%%%%%%%%%%%%%%%%%%%%%%%%%%%%%%%%%%%%%%%%%%%%%

\section{Languages}

\begin{tabular}{rl}
  English: & TOEFL ibt 106 (reading 29 $\cdotp$ listening 27 $\cdotp$ speaking 23 $\cdotp$ writing 27) \\
  Chinese: & Native Speaker \\
  German: & Basic \\
  French: & Basic \\
\end{tabular}

\section{Miscellaneous}

\begin{tabular}{rl}
  Interests: & Snowboarding $\cdotp$ Classical Guitar \\
  GitHub: & \href{https://github.com/hwp}{github.com/hwp} (highlighted projects: \href{https://github.com/hwp/apkit}{apkit}, \href{https://github.com/hwp/notGHMM}{notGHMM}) \\
\end{tabular}

\end{document}

